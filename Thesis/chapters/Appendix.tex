\chapter{Example of Appendix Chapter} \label{appendix:example}

\begin{example}[Figure Caption Tweaking]
    Now we will show off some figures with tweaked position/extent of the captions.
    \Cref{fig:parallel-plate-capacitor} has a side-caption, while \Cref{fig:tighter_caption} has a caption that spans just the width of the figure.
    This utilizes the \package{floatrow} package and is inspired by the ITT \LaTeX{} template \autocite{ITTtemplate}.
\end{example}

\begin{figure}[!ht]
    \exportInkscapeSVG{parallel-plate-capacitor}% export already here, such that we can use it in the caption
    \fcapside[\FBwidth]{%
        \caption[Example of a figure with a side-caption.]{
            Example of a figure with a side-caption.

            It displays the two-dimensional electric field near one end of a parallel plate capacitor.
            \vspace{1ex}

            \textbf{\textsf{Legend:}}\\[0.4ex]
            \begin{tblr}{stretch=0.8, column{1} = {rightsep=2pt, cmd=\raisebox{-0.31ex}}}
                \adjincludegraphics[width=0.6\textwidth, Clip={0.87\width} {0.483\height} {0.0\height} {0.483\height}]{parallel-plate-capacitor} & equipotentials  \\
                \adjincludegraphics[width=0.6\textwidth, Clip={0.35\width} {0.068\height} {0.52\width} {0.898\height}]{parallel-plate-capacitor} & field lines     \\
                \adjincludegraphics[width=0.6\textwidth, Clip={0.20\width} {0.313\height} {0.67\width} {0.653\height}]{parallel-plate-capacitor} & capacitor plate \\
            \end{tblr}%
        }%
        \label{fig:parallel-plate-capacitor}%
        \floatfoot{You can also optionally use a footnote\\ for the figure caption.}%
    }{\includeInkscapeSVG[0.6\textwidth]{parallel-plate-capacitor}}%
\end{figure}
\begin{figure}[!ht]
    \ffigbox[\FBwidth]{%
        \caption{Example of a figure with a caption spanning just the width of the figure.}%
        \label{fig:tighter_caption}%
    }{\includegraphics[width=0.6\textwidth]{example-image-a}}
\end{figure}

\clearpage
\begin{example}[Multi-Paragraph Figure Caption with Verbatim Text]
    It is possible to have multi-paragraph captions for figures.
    One must remember to provide a short description as \macro{\caption[This is a Short Description]{...}}, or else \LaTeX{} will complain.

    See \Cref{fig:multi-paragraph_verbatim_figure} for an example, demonstrating also a workaround for typesetting verbatim text in contexts where \enquote{fragile} commands are not allowed.
\end{example}
\begin{figure}[!ht]
    \includegraphics[width=0.6\textwidth]{example-image-b}
    \caption[Necessary to provide short description.]{
        Example of a figure with a multi-paragraph caption.

        Notice the spacing between the paragraphs.
        It was customized using the \fakemacro{parskip} key in \fakemacro{\captionsetup} provided by the \package{caption} package.

        To typeset verbatim text in the caption, use the \fakemacro{\fakeverb{...}} command instead of the usual \fakemacro{\verb|...|}, which is not allowed in captions.
    }%
    \label{fig:multi-paragraph_verbatim_figure}%
\end{figure}

\section{Appendix Section}%
\label{sec:Appendix Section}

Note the numbering of various environments in the appendix.

\begin{definition}[Math in the Description --- \(\sin(\alpha)\approx\alpha\)]
    This is an example definition in an Appendix.
    Note the automatic switch to the alternative sans math font in the Definition description.
\end{definition}

\begin{remark}
    The page header reflects that this is an appendix page.
\end{remark}

\begin{example}[Equation Numbering and Referencing]
    As was mentioned already in \Cref{sec:Document Structure}, equations share numbering with \emph{structure environments}.
    For example, the equation
    \begin{equation} \label{eq:appendix_equation}
        \phi^{*}\bm{g}' \overset{!}{=} \Omega^{2}\bm{g} \equiv \E*^{2\omega}\bm{g}
    \end{equation}
    is numbered as \eqref{eq:appendix_equation} in the appendix.

    We can reference this equation using \custommacro{\Cref} as \Cref{eq:appendix_equation}.
    Starred variant \custommacro{\Cref*} results in \Cref*{eq:appendix_equation}.
    If you desire less verbose output, you can use \macro{\eqref}, which gives \eqref{eq:appendix_equation}.
\end{example}

\begin{theorem}[Example with Math at the End]
    Theorem ending with math, with proper spacing by utilizing \custommacro{\qedhere} (can even use an optional argument to finetune the spacing)
    \[
        a^{2}+b^{2}=c^{2} \eqend \qedhere
    \]
\end{theorem}
