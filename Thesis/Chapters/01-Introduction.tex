\documentclass[../Main.tex]{subfiles}
\begin{document}

This report presents the outcomes of a research internship conducted as part of the final year project at INSEA. It focuses on advancing coordination efficiency in cooperative multi-agent reinforcement learning (MARL), a domain increasingly critical in real-world applications such as robotics, traffic management, and distributed control systems. Motivated by the scalability challenges faced by existing state-of-the-art algorithms, particularly QMIX, this work investigates novel mechanisms to reduce unnecessary coordination and computational overhead. The project is rooted in both theoretical insights and empirical validations, combining rigorous analysis with practical experimentation.



This report is structured as follows:
\begin{itemize}
    \item \textbf{Chapter 1} provides the institutional context of this research, detailing the mission and objectives of Mohammed VI Polytechnic University and the AI Movement Center.
    \item \textbf{Chapter 2} establishes the theoretical foundations, covering the principles of single-agent and multi-agent reinforcement learning, and provides a detailed analysis of the QMIX algorithm.
    \item \textbf{Chapter 3} presents our experimental methodology and results, including the design of QMIX-Masked, the experimental setup, and a rigorous analysis of its performance.
    \item \textbf{Chapter 4} concludes the report by summarizing our key findings, discussing their implications, and outlining promising directions for future work.
\end{itemize}

\end{document}
