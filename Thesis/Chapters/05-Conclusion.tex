\documentclass[../Main.tex]{subfiles}
\begin{document}

This report has presented an empirical investigation into the scalability and coordination mechanisms of value-based multi-agent reinforcement learning. Motivated by a theoretical analysis of the QMIX algorithm's architectural bottlenecks, we questioned the efficiency and necessity of its rigid, global coordination structure.

\section{Summary of Findings}

The central finding of this research is the \textbf{remarkable robustness of the QMIX algorithm to random information pruning}. Through a series of controlled experiments on the StarCraft Multi-Agent Challenge, we demonstrated that applying a random mask to as much as 60\% of the agent inputs before the mixing stage resulted in no statistically significant degradation in performance on our primary test scenario.

This result strongly suggests that the global, fully-connected coordination mechanism employed by QMIX contains significant redundancy. The system is capable of learning effective policies even when a large portion of inter-agent information is unavailable at any given step, indicating that forcing full, global coordination is not always essential for achieving high performance in complex cooperative tasks.

\section{Limitations}

While our findings are compelling, it is important to acknowledge the limitations of this study. Our primary analysis was conducted on a single, medium-difficulty SMAC map (\texttt{2s\_3z}) to ensure a stable and reliable baseline. The robustness observed may not generalize to all scenarios, particularly those requiring extremely tight and complex coordination among all agents simultaneously.

Furthermore, the random masking strategy used in our experiments is intended as a diagnostic tool, not as a proposed optimal algorithm. Its purpose was to probe the system's properties and test a hypothesis, rather than to enhance performance directly.

\section{Future Work}

Our results open up several exciting and promising directions for future research. The key takeaway is that selective coordination is not only viable but desirable. The most promising path forward lies in replacing our simple random masking with \textbf{intelligent, learned coordination mechanisms}. Future work could explore:

\begin{enumerate}
    \item \textbf{Attention-Based Masking:} Implementing an attention mechanism that allows agents to learn which teammates to coordinate with at each timestep, based on the current state. This would create a dynamic and sparse coordination graph, improving both efficiency and performance.

    \item \textbf{Graph Neural Networks (GNNs):} Using GNNs to explicitly model the relationships between agents. This would allow value information to be passed only between relevant neighbors, effectively learning the structure of the coordination graph in a more principled way.
    
    \item \textbf{Broader Empirical Validation:} Extending this study across a wider range of difficult SMAC maps and other MARL benchmarks to determine the generality of the observed robustness.
\end{enumerate}

Ultimately, this work serves as a foundational step toward a new class of scalable, efficient, and adaptive multi-agent learning algorithms. By moving away from rigid global architectures, we can unlock the potential of MARL to solve even larger and more complex real-world problems.




\biblio % Needed for referencing to working when compiling individual subfiles - Do not remove
\end{document}