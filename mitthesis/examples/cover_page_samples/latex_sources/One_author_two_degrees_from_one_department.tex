% !TEX TS-program = LuaLaTeX

% Author: John H. Lienhard (c) 2025. Reuse under the MIT license: https://ctan.org/license/mit 
% Version 1.03, 2025/05/01

% Documentation: https://ctan.org/pkg/mitthesis

%% Don't modify the \DocumentMetadata command unless you know what it does. 
%% If this command throws an "undefined" error, your latex system is out of date: try commenting this command out.
\DocumentMetadata
{
	testphase={phase-III,math,table,firstaid,title},
	pdfstandard = ua-2,
	pdfversion  = 2.0,
	lang		= en-US,
}
%%%%%%%%%%%%%%%%%%%%%%%%%%%%%%%%%%%%%%%

\documentclass[]{mitthesis}

\usepackage{microtype}


%%%%%%%%%%%  Metadata  %%%%%%%%%%%%%%%%%%%%%%%%%%%%%%%%%%%%%%%%%%%%%%%%%%%%%%%

% Most of the document metadata is created automatically. 
% The following items should be adjusted to match your work. <================= !!!!!!!!!!

\hypersetup{%
	pdfsubject={Template for writing MIT theses with the mitthesis class},
	% Change this to briefly state topic of your thesis 
% 
	pdfkeywords={Massachusetts Institute of Technology, MIT},
	% Add keywords that will help search engines and libraries to find your work.
	% Includes the name[s] of the author[s] 
	% (If you have used \DocumentMetadata, at line 15, you can just put "\CopyrightAuthor," for the names.)
%
	pdfurl={},
	% If you have a url for the thesis, put it here. Otherwise delete this.
	% (MIT Libraries will put your thesis in DSPACE with a persistent url after you submit it.)
%	
	pdfcontactemail={},
	% You can put a [permanent] email address into the metadata, if you like.
	% Otherwise delete this.
%
	pdfauthortitle={},
	% If you have a title, you can include it here.
}

%%%%%%%%%%%%%%  End preamble %%%%%%%%%%%%%%%%%%%%%%%%%%%%%%%%%%%%%%%%%%%%%%%%%%%%%%%%%%%%%%%%%%%%%

%%%%%%%%%%%%%%%%%%%%%%%%%%%%%%%%%%%%%%%%%%%%%%%%%%%%%%%%%%%%%%%%%%%%%%%%%%%%%%%%%%%%%%%%%%%%%%%%%%

\begin{document}


%%% edit the following commands to match your thesis %%%%%%%%%%

\title{The Atomic Theory as Applied To Gases, with Some Experiments on the Viscosity of Air}

% \Author{Author full name}{Author department}[Author's first PREVIOUS degree][Author's second PREVIOUS degree][...
% Note that third, fourth, fifth, and sixth arguments are optional [] and may be omitted

% note on names: most of the following names are made up; Silas Holman was a professor at MIT in the 19th century.

\Author{Silas W. Holman}{Department of Physics}%[B.S. Physics, MIT, 1876][MBA, Ferengi School of Management, 2022]
%\Author{Luisa Hernández}{Department of Research}[B.S. Mechanical Engineering, UCLA, 2018][M.S. Stellar Interiors, Vulcan Science Academy, 2020][MBA, Ferengi School of Management, 2022]

% Use once for each degree fulfilled by thesis
% For two degrees from one department, leave the department argument blank for the second degree {}.
\Degree{Bachelor of Science in Physics}{Department of Physics}
\Degree{Master of Science in Physics}{}
%\Degree{Bachelor of Science in Mechanical Engineering}{Department of Mechanical Engineering}

% If there is more than one supervisor, use the \Supervisor command for each.
\Supervisor{Edward C. Pickering}{Professor of Physics}
%\Supervisor{Secunda Castor}{Professor of Research}
%\Supervisor{Quintus Castor}{Professor of Log Dams}

% Professor who formally accepts theses for your department (e.g., the Graduate Officer, Professor Sméagol,...)
% If more than one department, use more than once
% If you need to reduce vertical space, put the acceptor title in the second argument and leave the third blank {}.
\Acceptor{Primus Castor}{Professor of Wetlands Engineering}{Undergraduate Officer, Department of Physics}
%\Acceptor{Tertius Castor}{Professor of Log Dams}{Graduate Officer, Department of Research}
\Acceptor{Quarta Castor}{Professor of Lodge Building}{Graduate Officer, Department of Physics}

% If your title page is overflowing (from too many names, degrees, etc.), you can scale 
%    down the Signature block at the bottom with this command, or use another creative solution...
%\SignatureBlockSize{\small} %\SignatureBlockSize{\footnotesize}

% Usage: \DegreeDate{Month}{year}
% Valid degree months are September, February, or June.  
\DegreeDate{June}{1876}

% Date that final thesis is submitted to department
\ThesisDate{May 18, 1876}


%%%%%%  Choose whether to have a CREATIVE COMMONS License  %%%%%%%%%%%%%%%%%%%%%%%%%%%%%%%%%%%%%%
%
% If you are using a cc license, put details of your cc license here. 
% Omit this command if you are not using a cc license.
%
\CClicense{CC BY-NC-ND 4.0}{https://creativecommons.org/licenses/by-nc-nd/4.0/}
%
%%%%%%%%%%%%%%%%%%%%%%%%%%%%%%%%%%%%%%%%%%%%%%%%%%%%%%%%%%%%%%%%%%%%%%%%%%%%%%%%%%%%%%%%%%%%%%%%%

%%% Titlepage
\maketitle

% The abstract environment creates all the required headers and footnote. 
% You only need to add the text of the abstract itself in the file abstract.tex
\begin{abstract}
	% From mitthesis package
% Version: 1.01, 2023/06/19
% Documentation: https://ctan.org/pkg/mitthesis
%
% The abstract environment creates all the required headers and footnote. 
% You only need to add the text of the abstract itself.
%
% Approximately 500 words or less; try not to use formulas or special characters
% If you don't want an initial indentation, do \noindent at the start of the abstract

The developments of the ``kinetic theory'' of gases made within the last ten years have enabled it to account satisfactorily for many of the laws of gases. The mathematical deductions of Clausius, Maxwell and others, based upon the hypothesis of a gas composed of molecules acting upon each other at impact like perfectly elastic spheres, have furnished expressions for the laws of its elasticity, viscosity, conductivity for heat, diffusive power and other properties. For some of these laws we have experimental data of value in testing the validity of these deductions and assumptions. Next to the elasticity, perhaps the phenomena of the viscosity of gases are best adapted to investigation.\footnote{Text from Holman (1876): \doi{10.2307/25138434}.}  
% in this case, use \input rather than \include because you are inside an environment
\end{abstract}

\end{document}	
